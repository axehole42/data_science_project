\documentclass{article}
\usepackage{geometry}
\geometry{a4paper, margin=1in}
\usepackage{hyperref}

\title{Predicting Net Income Direction Using Financial Ratios}
\author{Data Science Project Team}
\date{\today}

\begin{document}

\maketitle

\section{Introduction}
This document records the reasoning, methodology, and progress of our data science project. The primary goal is to predict whether a firm will report positive or negative net income in the next fiscal year ($t+1$) based on financial data from the current year ($t$).

\section{Research Question}
\textbf{Question:} Can we predict whether a firm will report positive or negative net income next year ($t+1$) using current-year ($t$) financial ratios and features?

\begin{itemize}
    \item \textbf{Target Variable:} Binary classification. 1 if Net Income Adjusted (\texttt{niadj}) $> 0$, else 0.
    \item \textbf{Scope:} Compustat data (North America), 2014--2024.
\end{itemize}

\section{Data Cleaning and Preprocessing}
\subsection{Initial Inspection}
The raw dataset contains financial indicators identified by \texttt{gvkey} (company) and \texttt{fyear} (fiscal year).

\subsection{Cleaning Steps}
\begin{itemize}
    \item \textbf{Sorting:} Data is sorted by \texttt{gvkey} and \texttt{fyear} to ensure correct time-series alignment.
    \item \textbf{Duplicates:} Rows with duplicate \texttt{gvkey}-\texttt{fyear} pairs are removed to ensure uniqueness.
    \item \textbf{Date Parsing:} \texttt{datadate} is converted to a standard datetime format.
\end{itemize}

\section{Feature Engineering}
\textit{To be populated...}

\section{Model Selection}
\textit{To be populated...}

\end{document}
