\documentclass{article}
\usepackage{geometry}
\geometry{a4paper, margin=1in}
\usepackage{hyperref}

\title{Predicting ROA Improvement Using Financial Ratios}
\author{Data Science Project Team}
\date{\today}

\begin{document}

\maketitle

\section{Introduction}
This document records the reasoning, methodology, and progress of our data science project. The primary goal is to predict whether a firm's **Return on Assets (ROA)** will **improve** in the next fiscal year ($t+1$) compared to the current year ($t$), based on financial data from the current year.

\section{Research Question}
\textbf{Question:} Can we predict whether a firm's profitability will improve or deteriorate in the next fiscal year ($t+1$) using current-year ($t$) financial ratios?

\begin{itemize}
    \item \textbf{Target Variable:} Binary classification of directional change.
    \item \textbf{Formula:} $y_{i,t} = 1$ if $ROA_{i,t+1} > ROA_{i,t}$ (Improvement), else $0$ (Deterioration).
    \item \textbf{Metric:} Return on Assets ($ROA = NI / AT$) is used to control for firm size.
    \item \textbf{Scope:} Compustat data (North America), 2014--2024.
\end{itemize}

\section{Data Cleaning and Preprocessing}
\subsection{Initial Inspection}
The raw dataset contains financial indicators identified by \texttt{gvkey} (company) and \texttt{fyear} (fiscal year).

\subsection{Cleaning Steps}
To ensure data quality and relevance, the following filtering steps are applied sequentially:

\begin{enumerate}
    \item \textbf{Standard Compustat Filters:}
    \begin{itemize}
        \item \textbf{Industry Format (\texttt{indfmt}):} Kept only 'INDL' (Industrial) to exclude financial services which have different accounting structures.
        \item \textbf{Data Format (\texttt{datafmt}):} Kept only 'STD' (Standardized) for consistency.
        \item \textbf{Population Source (\texttt{popsrc}):} Kept only 'D' (Domestic/US) to focus the scope.
        \item \textbf{Consolidation Level (\texttt{consol}):} Kept only 'C' (Consolidated) to avoid double-counting via parent companies.
    \end{itemize}
    
    \item \textbf{Missing and Zero Values Filter:}
    \begin{itemize}
        \item Rows with missing or zero \textbf{Total Assets (\texttt{at})} are dropped. Assets must be strictly positive to serve as a denominator for financial ratios (preventing division by zero).
        \item Rows with missing \textbf{Net Income Adjusted (\texttt{niadj})} are dropped. Note that \textbf{negative values of \texttt{niadj} are retained}, as predicting negative income is the core objective.
    \end{itemize}

    \item \textbf{Duplicate Removal:}
    \begin{itemize}
        \item Sorted by \texttt{gvkey} and \texttt{fyear}.
        \item Dropped duplicates, keeping the most recent update based on \texttt{datadate}.
    \end{itemize}
    
    \item \textbf{Type Standardization:}
    \begin{itemize}
        \item Converted mixed-type columns (e.g., prices with commas) to proper numeric formats.
    \end{itemize}
\end{enumerate}

\section{Feature Engineering}
\subsection{Constructed Ratios}
We constructed 13 financial features based on literature standards, categorized into:
\begin{itemize}
    \item \textbf{Profitability:} ROA, OCF/Assets, Accruals.
    \item \textbf{Liquidity:} Current Ratio, Cash Ratio, Working Capital/Assets.
    \item \textbf{Leverage:} Financial Debt/Assets, Debt/Equity, Total Liabilities/Assets.
    \item \textbf{Growth/Size:} Asset Growth, Log(Assets).
    \item \textbf{Trends:} $\Delta$ ROA, $\Delta$ Leverage, $\Delta$ Current Ratio.
\end{itemize}

\subsection{Data Quality Handling}
\begin{itemize}
    \item \textbf{Division by Zero:} Ratios resulting in infinity (e.g., when Current Liabilities = 0) were replaced with NaN (Missing) to prevent model instability.
    \item \textbf{Missing Debt:} Missing values for Long-Term Debt (\texttt{dltt}) and Debt in Current Liabilities (\texttt{dlc}) were filled with 0, under the assumption that non-reporting implies no interest-bearing debt.
    \item \textbf{Time Alignment:} Trend features and the Target Variable were only calculated where strict $t$ and $t+1$ (or $t-1$) fiscal years were available.
    \begin{itemize}
        \item \textbf{Target Generation Drop:} Approximately 12,000 rows were dropped. This includes the \textbf{last available year of data for every firm} (e.g., 2024), as the future outcome ($t+1$) is unknown and thus cannot serve as a training label. It also includes rows preceding a gap in reporting (e.g., 2012 data followed by 2014).
    \end{itemize}
\end{itemize}

\section{Model Selection}
\textit{To be populated...}

\end{document}
